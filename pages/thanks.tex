% 致谢页

\clearpage
\phantomsection
\addcontentsline{toc}{chapter}{致谢}

\chapter*{致谢}
余撰此文时,春之珞珈或将已矣。月初,余犹叹未有赏樱之缘,月末却复恨
无与折柳之机。

  缘何至此?己亥岁除,荆楚大疫,逾翌年春,染者万计。一时之间,通衢空
继踵之城;四海之内,惶惶闻九州之野。此诚危急存亡之秋也。然,壮士出于兵
破士北,贤明起于大树将颠。乃见国士连夜北上,将扣鲸敌所在;亦闻白衣一苇
渡江,欲治伤痍之重。青丝、白发皆身先士卒,商贾、布衣尽争先解囊。盖闻举
国一心,亦若此也。六旬余,大疫终有治,山河还无恙。此当敬谢天下为先。

而值次大疫,求学之旅即止,每自苦读中惊觉,却恍若隔日。巍巍珞珈,百
年黉门,余求学于此已近四秋矣。惟叹时光不老,韶华难留。余平民世家,聿修
祖德,孝悌累洽,父严母慈。自求学起,徒养余求学之路,予余心存归处,不思
回报,父母之督察,为余顺之成文大有助力。双亲鬓渐发白,然大爱不曾稍减,
于我备至更加。余恨不能为其分忧,不能为其担责,甚为内疚,此余跪而叩谢者
一也。来日,当益勉之学、工作,不负父母谓余之殷殷期!

  恩师李先生晓雷,导余于狭路,示余以通途。本论文之撰写,自题目遴选至
研究思路,自框架结构至细枝末节,皆得先生悉心指点。感荷先生拳拳之心,念
先生之恩重,谢无疆焉!今虽将辞,当不忘师恩,精进图强,以期不失其望也!

  恰同学少年者,皆四海之菁英也。逢风华正茂之时,得遇同窗之谊、金兰之
交、共渡之缘,幸甚至哉!侣缘者谁?

  临书仓卒,谨申数字,用展寸诚,祈恕不恭。书之有尽,敬谢难穷;感慨惶
恐,不知所云。遂以拳拳之心,对吾之恩师、椿萱及列位亲友再致谢忱,