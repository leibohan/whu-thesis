% 致谢页

\clearpage
\phantomsection
\addcontentsline{toc}{chapter}{致谢}

\chapter*{致谢}
尽管这不是我第一次写论文,但是这一次写论文,应该是我第一次独自一人完成一篇正式的论文。在参与一篇论文的写作时,写论文是一项团结的工作,我需要做好我的部分,并且可以在小组开会的时候与他人分享、讨论我的工作,总能在想不明白的地方前进下去;但是独自一人写论文,要面对一定的孤独,要一个人解决程序中的问题、自学以前不懂的知识,因为这就是学位论文,尤其是疫情期间的学位论文。

这一次的毕业季是不同的,后无来者尚不敢断言,但是果真前无古人。借助反向代理工具,远程使用几跳之外的服务器,无疑对我的实验产生了巨大的影响,但是作为一个计算机专业学生,我的骄傲让我只能拼命想办法完成我的实验,尽量不要缩减题目预设的要求。毕竟,这是独一无二的本科毕设。还好,我的孤独只是相对的,因为有太多可爱的人陪着我,在我一度对实验绝望的时候用自己的经验帮我答疑解惑。

首先我要感谢我的父母。这次进行毕业设计期间,我很幸运地与父母朝夕相伴。母亲刘雯老师从一个硕士生导师的身份,为我的论文写作提了很多建议,让我有了独自完成它的信心。父母在我进行毕设期间,尽可能多地承担起了买菜、做饭、洗碗等家务,让我专心进行毕设;有的时候,我因为要尽快进行实验,而需要连续工作、作息颠倒,在家的彻夜工作可能打扰到了父母的休息,再次真的要感谢父母的包容。

同时,我也要感谢我的姥姥。姥姥在家里,身体越来越差,却在每年过年的时候都给我做黏米饭,暑假的时候给我做红烧肉。每一次我回家,姥姥都要问我,成绩怎么样;而我,也愿意把学校里每一个有趣的瞬间,与姥姥分享。就在去年,姥姥不在了,我习以为常的一个锚点,似乎突然撒开了。我还记得,每次我回家的时候,姥姥一定要给我塞几百块钱,说是谢谢我来看她,是给我的路费。您不知道,我有多么想您,是您二十余年如一日的爱,让我不忘初心,奋勇前行,在学习的道路上越走越远。姥姥,我毕业了,您看到了吗?谢谢您!

第二要感谢我这一篇论文的导师:黄浩老师和庄越挺老师。黄浩老师在本科期间带领我进行科研,让我对科研有了一定的认识,并且帮助我获得了宝贵的文献阅读、论文撰写与专利撰写的经验,让我在这一次的毕业设计中可以快速上手,也让我对科研充满兴趣。庄越挺老师收我为徒,接纳我在博士研究生阶段继续深造,也给我设定了这个题目,让我把本科阶段的句号作为研究生阶段的破折号,让我初步接触研究生阶段的研究领域,通过一次完整的实践让我在研究生阶段快速起步。希望我未来的学习研究之路也能顺利、多出成果。两位老师在本篇论文的撰写中,也给了我很多帮助。

同时,也感谢计算机学院的各位任课老师,蔡朝晖老师的计算机组成原理课深入浅出,让我轻松掌握,有了自信;梁超老师辅导我们参加了多项比赛,获得了优异的成绩;董文永老师用他独特的算法思维,让我们进一步掌握了与普通编程不同的算法的思考模式……谢谢老师们对我的教导,我争取活学活用,在未来的科研中让知识开花。

第三,我要感谢母校武汉大学,尤其是弘毅学堂。是学堂和学校的各位领导和辅导员老师支持我们走的更远、尝试更多。在学堂的资助下,我参加了多项比赛,还有一次出国交流学习,都是学堂和学校负担了费用,才让我没有后顾之忧,获得了比区区费用远远珍贵的东西。

最后,要感谢我亲爱的同学们。我的室友李翌明、原昊博、施仲翼,我的同学陈子轩、周稚璇、田雅婷、彭凯飞以及班里的每一位同学,其他专业的同学眭策、麻尧崴、韦钰、李海博、薛睿哲、邢斌,学弟学妹李蕴哲、毛文月、王嘉庆,科幻协会的诸位同学、高尔夫球协会的各位学长、咖啡协会的各位成员、弘毅桌游社的各位同仁,校外在专业上帮助我的史灿阳、龙洲鸣、买明、张兴凤、赵秉坤。你们在我本科生活中增添了亮丽的色彩,不仅在我学习遇到困难时帮助我答疑解惑,还带我参加了各种比赛、让我了解了各种课堂外的知识。你们给我的记忆是我大学生涯里最宝贵的财富,希望我们的友谊长存,在日后也能互帮互助,携手成长。

大学四年,真的是很长的一段时间。有幸,在武汉大学度过的这段经历,有惊无险、多姿多彩。临到离别之际,多少有些慨叹,多少有些不舍,但是感谢这四年的青春,我无怨无悔,有了切实的成长,经历了切实的岁月。

\vspace{2em}
\flushright{雷伯涵\quad\hspace{5em}}
\flushright{二〇二〇年四月\quad}
\flushright{身在家心在珈\quad\hspace{0.5em}}