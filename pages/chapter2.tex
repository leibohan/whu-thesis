% 文中引用说明

\chapter{相关技术理论}

\section{GAN生成模型选取使用}

\subsection{衍生模型分类与特点}
目前对抗生成网络的衍生模型众多,其优化方式大抵由两个大方向衍生而来。一个方向是由损失函数的改变来优化GAN模型的效果,另一个方向则是从模型使用的角度来优化GAN模型的效果。在表~\ref{tab:1.1}中,列举了一些最为常见的GAN模型衍生模型。

\begin{table}[!htb]
    \centering
    \caption{}
    \label{tab:1.1}
    \begin{tabular}{cccc}
        \toprule
        从损失函数角度&\multicolumn{3}{c}{从模型应用角度提出的优化GAN模型}\\
        \cline{2-4}
        提出的优化GAN模型\upcite{fgans}&网络构架角度\upcite{mirza2014conditional}&编码器角度&其他角度改进\\
        \hline
        \multirow{5}{0.3\textwidth}{Least Square GANs, Loss-Sensitive GAN, Fisher GAN, WGAN, WGAN-GP, WGAN-LP, f-GANs\upcite, DRAGAN等}&\multirow{5}{0.19\textwidth}{CGAN, DCGAN, InfoGAN, StackGAN, AL-GAN等}&\multirow{5}{0.19\textwidth}{BEGAN, VAE-GAN, tDCGAN, BiGAN,文献中的算法\upcite{编码器GAN1, 编码器GAN3, 编码器GAN2}等}&\multirow{5}{0.19\textwidth}{LAPGAN, ESRGAN, SRGAN, 3D-GAN, MGAN等}\\ \\ \\ \\ \\
        \bottomrule
    \end{tabular}
\end{table}

图像生成的模型与基于网络构架优化的GAN网络模型最为贴合,本文设计使用StackGAN作为基础,并针对相关实现进行优化。

\subsection{StackGAN的特性}

\section{神经网络技术}

\subsection{循环神经网络}
