% 中英文摘要

\begin{cnabstract}{推荐算法;深度学习;个性化推荐;DeepFM}
  随着互联网的日渐普及,数据积累越来越丰富,已出现过载的现象。如何合
理有效地挖掘数据的潜在价值,以辅助推动经济增长,是当前的一个热门话题。

  相比之需要用户抱有明确目的性去筛选信息的搜索引擎,推荐技术可以借助
用户人口统计学属性和其交互行为等数据进行相关分析,进而实现自动地为用户
推荐他们需求的信息的效果。鉴于此,推荐技术在近些年来一直是研究热点,其
也随着时代日新月异。

本文基于推荐算法研究的考量,结合博物馆观众服务质量有待提高的需求,
深入探究了基于因子分解机的神经网络模型框架。并基于此框架和深度神经网络 构建了 DeepFM-D 模型,实现了一种面向博物馆观众服务的个性化推荐。

本文对课题实验结果作了相关分析,从预测评分准确性和推荐列表输出准确 
性两个角度出发,利用 FM、DNN 等不同算法模型在此评估标准上与 DeepFM-D 模型进行了推荐对比实验。实验结果表明,DeepFM-D 模型评估指标具有较明显 的优势,其效果的有效性得到验证。
\end{cnabstract}

\begin{enabstract}{Recommendation Algorithm; Deep Learning; Personalized Recommendation; DeepFM}
With the increasing popularity of the Internet, the accumulation of data is becoming more and more abundant, and there has been an overload phenomenon. How to mine the potential value of data reasonably and effectively to help promote economic growth is currently a hot topic.

In contrast to search engines that require users to have a clear purpose to filter information, recommendation techniques can use data such as user demographic attributes and their interactive behaviors to perform relevant analysis, thereby achieving the effect of automatically recommending the information they need for users. In view of this, recommendation technology has been a research hotspot in recent years, and it also develops rapidly with the times.

In this paper, based on the consideration of the research of recommendation algorithms, combined with the needs of museum audience service quality to be improved, the neural network model framework based on factorization machine is deeply explored. Based on this framework and deep neural network, the DeepFM-D model is constructed to realize a personalized recommendation for museum audience services.

This paper makes a relevant analysis of the experimental results of this subject. From two perspectives of the accuracy of prediction scores and the accuracy of the output of the recommendation list, different algorithm models such as FM and DNN are used to compare and recommend the DeepFM-D model on this evaluation standard. The experimental results show that the evaluation index of DeepFM-D model has obvious advantages, and the effectiveness of its effect is verified.

\end{enabstract}
