% 中英文摘要

\begin{cnabstract}{图片字幕;图片生成模型;生成对抗网络;长短期记忆}
  近年来随着深度学习技术的不断发展,计算机视觉和自然语言处理成为了计算机科学与人工智能交叉领域研究热点。其中图片和语言之间的翻译技术 是人类可以直接利用的技术,包括图片字幕技术和文本生成图片的系列技术,解决这一问题需要计算机在某种程度上分拆图片和语言中的信息,通过一定的规则,智能地筛选其中的关键信息,并说出他们之间的关联。图片翻译文本技术,可以被用来处理海量图片信息,从整体描述他们的主要内容,辅助人类理解画面;而文字生成图片技术,则可以帮助人们直观地直到自己表述话语的意义,并借此减少交流中出现出现的偏差。因此,利用快速的图片翻译文本技术和文本生成图片技术来制作一个一般人可以轻易使用的软件具有重要意义。

  本文通过调研,提出并设计了一套基于 技术的自然语言与图像互译系统,通过开发软件时训练好的相关模型,生成另一种格式的信息。本文的工作主要包括:

  (1)对输入图片的处理,使用简单的GCN网络从中提取特征向量,在LSTM网络中训练注意力机制函数的转移方法,用这一模型从图片中生成自然语言文本,实现图像向文本进行翻译的功能。

  (2)对输入文本的处理,使用NLP技术生成场景图,实现分三步的场景图像生成模型,并设计辨别器与之对抗,通过如此新建的GAN网络模型训练出对一般人来说更符合常理的复杂场景图像,实现文本向图像进行翻译的功能。

  (3)创新性地提出将两种用途近似、目标人群耦合的技术合并为一个完整可用的系统,方便普通人中有需求的群体便捷地使用到这两项技术。

  本文采用基于编码-解码结构的LSTM模型和基于影响力机制的GAN模型,分别复现了图像字幕技术和文本生成图像技术。从实验结果来看,本方案可以较好地实现文本和图像的互译功能。在后续迭代中,可以进一步提升数据集的丰富程度、针对模型的弱点加强训练,提高软件的翻译精度。
\end{cnabstract}

\begin{enabstract}{Image Captioning; Text to Image; Generative Adversarial Networks Model; Long-short Term Memory}
  With the development of deep learning technologies, computer vision and natural language processing have increasingly persuasively become a research hotspot in the crossing field of computer science and artificial intelligence. The technologies of translation between natural languages and images, including image captioning and image generative technologies based on texts, are for human beings' direct use. To implement the described function, it is necessary to extract and divide the features from the source format of information flows and generate an understandable version of target format information flow intelligently based on some specific rules. Image captioning can help users to process a huge amount of images and get the overall descriptions of them to help comprehension of the images, and text-to-image generation technology can help people to express there meaning with less deviation, on which people can rely to less the mistakes taking place during communications. Therefore, it is meaningful to make software to translate the images and texts to the other format readily with the two kinds of technologies for the people who need it.
  
  This paper proposes an image-language translation system based on investigation and research. By using the prepared model trained during the development of the software, the function of translation is realized. The main works in this paper are as follows:

  (1)Use the naïve GCN to extract the feature vectors from each image in the input dataset. Train the transmitting rules of attention areas under an LSTM model, and generate natural caption with the model to realize the translation from images to lines in natural language.

  (2)Process the input text with simple NLP dependency and generate a scene graph with the result. Realize a generative model based on three portions of the addition of objects and refinement, and design discriminators against them to produce a GAN model. Train a model with the GAN model to realize the translation from lines of text to images.

  (3)Revolutionarily propose to integrate the two technologies as an easy-to-use system for ordinary people because the two technologies are faced with combined groups of people and integrated software can make it easier to use the technologies for the people who need it.

  In this paper, an LSTM model based on decoder-encoder structure and a GAN model based on attention mechanism are used to recurrent the proper approaches of image captioning and text-to-image generating. From the experimental results, the project realized the translation between images and natural language scripts properly. In the later refinement works, we can improve the performance of the software by enriching the variety of datasets and training the model with the categories of examples with low precision.

\end{enabstract}
