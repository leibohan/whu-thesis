% Chapter 5

\chapter{总结与展望}
\section{设计意义}
这一毕业设计题目,是由指导教师通过对课题的理解划分,并结合我所掌握的知识和项目经验,在师生双方通过对题目研究范围、完成任务、达成目标等方面进行探讨后,确定的题目。通过这一设计,既开发了一个有应用价值的软件系统,也活学活用了软件工程、人工智能、机器学习等课程上学习到的知识,为研究生阶段的深造打下了基础。

\subsection{制作了一个简洁易用系统}
设计的成品软件,通过调用模型,可以实现预期的目标:实现自然语言与图像的互译。

图像标注生成文本的部分,使用了2015年Kevin Xu发表的图像字幕方法。这一方法的优势是,它通过编码-解码的方式和注意力机制函数,将图片提取成特征向量,并可以通过长短期记忆模型,生成通顺的简单语句,在实质上完成了图片向自然语言的翻译。

自然语言向图像翻译的部分,使用了李飞飞课题组Johnson在2018年发表在CVPR会议上的基于场景图的图像生成方法。这一方法基于有里程碑意义的StackGAN模型算法,又有了很大的进步。这一算法对生成器提出了很大的优化,使得模型可以在最终的生成图像中,生成很多的物体对象,并且保证它们形状与大小不会产生扭曲、变形,位置关系不会有巨大的谬误。它可以对复杂句说明的场景作出整体的图像生成,这是这个算法的先进性所在。

通过界面制作,将图像测试操作进行了简化,使这一技术应用到了普通人可以随意实用的平台上。

\subsection{活用知识、实践了深度学习技术}
这是我第一次独立一人从环境安装到训练出模型实践一个成熟深度学习算法的复现。

这次的设计使用了本科期间下列几门课程中学到的知识:《软件工程》、《机器学习》、《数据挖掘》等。我在这些课程中学到了深度学习的前置知识与基本概念,为这次成功的实践埋下了伏笔。

在博士研究生阶段,我将在人工智能领域继续探索,这一次的设计为我将来的学习与研究工作打下了扎实的基础。这次成功的实践,不但增强了我的自信,也让我多了一份深度学习项目实践的经验。

\section{未来发展方向}
这一设计还有很大的改进空间。

从界面上说,目前的功能有些单一,样式有些单调。未来可以将注意力信息也输入到界面显示上,为界面增加更多的趣味性与美观。

从图片标注上来说,目前的数据集虽然规模庞大,但是有些单一。经过更多的测试,可以发现对一些色块相似于训练集的图片,容易有过拟合的现象,比如将“road”识别为“river”,或者将建筑物上的花纹识别为“clock”。对于测试集中出现的注意力集中区域识别错误,应当增加相应的数据进行训练。

从文本生成图像上来说,目前的模型在蒙板内的生成物体图像还比较粗糙,另外图片整体的清晰度较低。在$64 \times 64$像素的清晰度下,小物体可能只占几百个或几十个像素点的区域,这样的情况下很难生成让人可以看懂局部的图像,这一点上人类没有判别器做的好,所以保险机制可能不是那么有效。未来可以对每一类别的物体专门训练插值算法模型,加大图像清晰度,让每一个蒙版内的物体图像内容对于自然人的可识别性更好。

%{\songti \bfseries 宋体加粗} {\textbf{English}}

%{\songti \itshape 宋体斜   体} {\textit{English}}

%%%{\songti \bfseries \itshape 宋体粗斜体} {\textbf{\textit{English}}}

%\section{编译}
%本模板必须使用XeLaTeX + BibTeX编译,否则会直接报错。 本模板支持多个平台,结合sublime/vscode/overleaf都可以使用。