% Chapter 5

\chapter{总结与展望}
\section{本文工作总结}

本文即从推荐技术研究的角度出发,结合提高智慧博物馆观众服务质量的理 念,深入探究了基于因子分解机的神经网络模型框架,并基于此框架和深度经网 络模型构建了 DeepFM-D 模型,实现了智慧博物馆观众服务推荐。本文首先分析 了推荐系统的国内外现状并对推荐算法做了相关研究。同时,基于目前博物馆存 在的观众服务问题,提出了相应的解决方案,实现了一种面向博物馆观众服务的 个性化推荐。主要工作如下:

1、数据处理方面: 首先对观众属性数据和行为数据以及活动数据进行了主要特征选取,在分析
数据过程中,发现部分数据有缺省、异常等情况,根据实际分析的需要,仔细考 虑后对数据进行去除异常值和归一化等处理,并对离散特征进行统一编码。并从 观众行为数据中提取出了观众对活动的隐式评分,从而更方便地展开推荐分析。 2、模型研究和实现方面:
对经典的推荐算法进行了博物馆推荐场景的相关研究,并基于深度学习的发 展对经典推荐算法的影响作了阐述。设计和搭建了基于博物馆观众数据的推荐模 型 DeepFM-D,并最终实现了基于博物馆观众属性数据向观众进行个性化推荐的 效果。
3、推荐结果评估和对比方面:
本课题根据研究实验内容从预测评分准确性和输出的推荐列表准确性两个角 度出发,对 DeepFM-D 模型进行了较为全面的评测。同时将 DeepFM-D 的推荐算 法与 FM、DNN、UserCF 和 CB 等不同的推荐算法模型进行了预测对比试验。实 验结果表明,DeepFM-D 模型评估指标具有较明显的优势,其效果的有效性得到 验证。
5.2 未来展望
  本文将博物观观众服务与推荐算法结合,实现了基于博物馆观众属性数据向
观众进行个性化推荐的效果,但缘于数据保密要求、实验研究时间以及技术水平
57
等限制,本研究课题还存在着很多可以改进的地方: (1)数据层面的拓展:
  本次基于博物馆观众行为数据的推荐算法研究受用户隐私保护政策调整、信
息保密措施和疫情防控等方面影响,研究数据较为单一,均为博物馆方提供的脱
敏后的数据库信息。且其中的观众行为数据较为稀疏,无法更好地去了解观众与
博物馆间实际的交互情况,无法更准确地分析观众的参观体验。未来可以在数据
层面上进行拓展,比如结合问卷调查、网络访问信息采集等数据,可以构建更多
更准确的特征,以提供更加个性化和智能化的推荐服务。并且观众隐式评分上没
有进行更多的函数选取和参数调整,未来可以考虑更多的观众行为特征,以更好
地刻画出观众活动隐式评分。
(2)算法层面的拓展
本研究的推荐算法基于 DeepFM 框架搭建,仅选择了前馈神经网络 DNN 模
型作为框架 Deep 部分。未来如果可以结合其他的深度学习网络模型增加 Deep 部 分的模型深度,同时利用可视化的方法来展现神经网络隐藏层每一层的学习情况, 可以帮助研究进行参数调优和实现模型结构的选取与调整。某种程度上,可以进 一步优化模型预测推荐的准确度。
(3)模型层面的优化 本研究课题的推荐算法研究均是基于离线场景进行的,并没有考虑模型性能,
即内存占用、处理时间等,同时研究数据均有延后性,没有考虑观众行为的实时 反馈。而在实际应用场景,模型训练效率要求较高。未来可以在模型层面进行优 化,如本研究实现的 DeepFM-D 模型,其主要计算来源于因子分解机部分的特征 组合乘法和神经网络部分的权重矩阵与辅助向量间的乘法,可以考虑利用多卡数 据并行、调节模型学习率和异步读取等手段对模型进行优化,同时,可以考虑加 入召回阶段,进行数据集的粗筛选,从而实现线上实时计算。