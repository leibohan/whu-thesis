%Chapter 1

\chapter{绪论}

\section{相关背景与需求}
随着社会发展与技术进步,我国目前越来越重视高效工具的研发。从2000年伊始到2010年互联网普及,再到2020年的现在,个人电子终端的功能已经越来越强大,每个人都需要更新期、更有用的效率工具。近五年,图片标注技术与生成模型都有爆炸式的发展;近两年,视觉问答技术(Visual Question Answering)\upcite{二号文章}、视频标注技术与秒级的图片理解更是让失明群体有了“读懂光芒”的希望,也让图片的理解从学术界或产业界的科研层面有了走向应用、走向市场的可能。

对于看不见的人来说,读懂视野这一技术是他们改善生活质量的工具,画出语言这一技术是他们表达自我的窗口;对更多的普通人群体来说,一个简洁易用的新奇效率工具,更是一种生活方式的改变,让更多的人通过机器的“魔力”,换一种角度来看语言和图片。

\section{技术应用意义与前景}

\subsection{研究意义}
图像-语义的双向翻译在目前已经有了很大的需求面。
最基本的就是,图像作为视觉信号,无法被视觉失能人群感知,可以将语义转化为听觉信号,方便视觉失能人群感知世界。更进一步,对于大量的图像信号,靠肉眼处理起来需要很大的人力成本,如果使用图像标注技术,则可以从图像中提取重要信息,对语义信息进一步处理,形成简报构成参考,辅助决策。
图片标注的主要参考意义在于全局的图片理解,这一点算法可能比人类做的更好,因为人类更可能只把注意力放在局部上,而忽略一些自己不关注的次要信息。


语义的图像化更是意义非凡。基础地说,同样面对失能人群,没有视觉的人通过这一系统也有了“创作”的能力,将他们的思想从肉体的局限中有限地拓宽了出口;或者说,即使是有视觉能力的人,如果不擅长绘画,也可以通过这一技术直接表达自己思想中的画面。更进一步地说,想象力弱是很多成年人能力的局限,而一个语义向图像的转化,则可以使得一个人的表述清晰、直观地呈现在他人的视野里,可以促进交流;而表述之人也可以根据可视化的表达,发现自己表达中的问题,及时予以修正,而这是我们生活中每个人都需要的。

可以说,生成图像不仅仅是作为观赏,它可以切真实际地改变我们的生活方式。

\subsection{研究前景}
现在并没有简单易用的商业系统,可以提供语义与图像互译的简便功能,大部分系统都只能作为技术的样品,做单向的翻译工作,目前主要在各大展览会上起到展示企业技术实力之用。对于常常需要与图片、交流打交道的人来说,这样一套系统对工作效率提高很多;广泛地,对于任何一个人,这类系统都可以改变他的生活方式。

\section{相关工作}
首先需要明确,根据目前自然语言处理技术状况,语义的标注和自然语言的生成在操作上的难度没有很大的区别。在设计中,我会使用自然语言作为交互的媒介,方便用户使用、直观感受技术的表现。

\subsection{图像翻译语义}
\subsubsection{图片标注研究现状}
在图片转化为文字工作的早期阶段,研究人用常用基于模式的方法来自动完成图像标注。在这种方法中,总是先捕捉图片中的各种可视信息,然后在固定的句式中填写这些信息,以填空的方法完成句子。用这种模型,很多算法都可以给出差强人意的输出语句,但是这样做的弱点是,输出遵循既定的模式和概念,输出的多样性很差,语言僵硬。

为了解决上述问题,有一些研究中使用了检索式方法来输出可变长度的语句\upcite{fang2015captions, lebret2015phrase}。这种方法是通过组合从图片中捕捉到的短语来输出图片标注结果。很明显,这种方式生成的语句的句式更为灵活,信息更为饱满,但是生成的语句一般限定于被检索的图片库,对一些特定的图片无法生成正确的结果。

近期由于机器翻译中的大进步,图像标注领域也开始应用了编码器解码的架构\upcite{我佛了这篇文章怎么了,you2016image,fang2018refining,zhu2018captioning}。这种框架可以运用深度神经网络来生成每张图片新的内容语义,比过去的方法更为准确;在这种框架下,图片被编码神经网络编码为中间表示形式,而可以用解码RNN网络将其直接翻译为自然语言。Vinyals提出一种神经图像标注器的方法(Neural Image Caption Generator,称作Google NIC)\upcite{vinyals2015show},可以提取目标图片的可视特征,用一个长短期记忆模型来得出图像标注。

\subsection{语义生成图像}
将语义转化为图像原本是画家的专利,现在运用模型也可以用计算机自动生成图像。生成式模型原本不能生成图像,到了GAN网络发展到一定程度,才出现了自动生成图像的概念。

生成方法、判别方法分别时机器学习的两大分支方向,而生成式模型则是用生成方法来生成样本的一类模型。传统的生成式模型设计比较简单。有一类生成式模型是从人类理解角度进行设计的,比如说最大似然估计法、近似法\upcite{kingma2013auto,rezende2014stochastic}与马尔可夫链法\upcite{hinton1984boltzmann, ackley1985learning, hinton2006reducing}等,这一类方法对于机器来说各有限制。最大似然估计法的参数更新直接受限于数据样本,数据样本不够丰富会限制生成模型的结果;近似法的目标函数太过复杂,算法只能逼近目标函数下界;马尔可夫链法的缺点便是复杂度过高。从机器理解角度设计的算法一般不直接进行拟合或者估计,而是通过采样数据样本调整模型,一般这种方法人类无法直接理解,但是生成样本是人类可以理解的。

主流的生成图像方法为GAN网络,即生成对抗网络(Generative Adversarial Networks),由由Goodfellow et al.在2014年首次提出\upcite{goodfellow2014generative}。目前,它已经发展成了生成式神经网络最大的热点,其研究得到了长足的发展。短短几年之内,已经有了一百余种GAN网络的衍生模型,其应用范围囊括了包括自然语言、图像处理、计算机视觉在内的各个领域。

生成式对抗网络的出现为计算机视觉应用提供了新的技术和手段,它以独特零和博弈与对抗训练的思想生成高质量的样本,具有比传统机器学习算法更强大的特征学习和特征表达能力。目前在机器视觉领域尤其是样本生成领域取得了显著的成功,是当前研究的热点方向之一。GAN的不同模型在生成样本质量与性能上各有优劣。当前的GAN模型在图像的处理上取得较大的成就,能生成以假乱真的样本,但是也存在网络不收敛、模型易崩溃、过于自由不可控的问题。

目前很多人用这一网络完成了有趣的小应用,包括模糊图片增加清晰度等,但是它最神奇的用法还是直接生成图像,用语义可以直接生成对辨别器不可分辨的图像。

\subsection{自然语言处理}
\subsubsection{背景介绍}
自然语言处理(Natural Language Processing, NLP)是指计算机对自然语言的处理算法。所谓自然语言,就是说人类在社会文明发展过程中为了交流而逐渐发展出的语言,例如中文、英文、日文,甚至包括手语,都是自然语言。因为自然语言之间的关系远远不是函数映射那么简单的事,所以这个领域经历很长的发展历程才到了今天相对成熟的地步。

自然语言处理的发展历程大致分了三个阶段。第一个阶段是上个世纪后半叶,
在二战之后洛克菲勒仅仅会的瓦伦·威佛等人在展望计算机技术的未来应用时,认为计算机可以用不同语言之间的简单词汇替换来完成翻译工作。在现在国际化知识丰富的我们来看,显然可以看出这种方法是行不通的,但是上世纪后期,
大量学者耗费人力物力,建立语言之间的词典,为每一个词汇做了映射,以实现计算机翻译。但是翻译结果并不理想,教训便是自然语言的理解需要考虑上下文关系。第二阶段是世纪之交的二十年。计算机技术迅猛发展,学者开始发现仅靠统计和替换无法完成翻译工作,而且神经网络也初步成熟,人工智能再次登上时代的焦点。例如,2005年,Pradhan\upcite{pradhan2005support}提出了语义角色解析的机器学习算法,使用SVM技术扩展了前人的工作\upcite{surdeanu2003using,gildea2002necessity},改进了数据的泛化功能。

第三阶段是近十年,大数据的积累帮助了自然语言处理的语料库丰富,促进了算法表现的提升;另一方面,音视频处理技术中深度学习的应用更为成熟了。其应用例如:2016年,Y Yao et al.\upcite{yao2016bi}提出了一种基于长短期记忆的中文分词方法,突破了上下文窗口大小限制,可以保存前文的重要信息;2015年Jie Zhou et al.\upcite{zhou2015end}提出了不使用解析的端到端SRL学习系统来分析语义,使用双向RNN,只使用文本作为输入功能,不使用语法知识。

\subsubsection{自然语言应用例子:IBM Watson}
2011年,在美国的一档综艺节目中,IBM公司云服务器bluemix平台上的Watson板块NLP技术相关产品大放异彩,当时因出色的对话表现爆红。

在IBM bluemix云平台上的Watson板块,有多语言对应的多种NLP相关应用,包括语音识别、语音生成等,经过实际试用,发现目前制作水平比较完善,但是中文包的翻译效果比较差。经过调研发现,中文的翻译训练语料库主要是使用专业文件书籍构成,导致语言不够丰富。这也给了我们一个教训,在实际NLP训练中,一定要使用比较丰富、贴近日常生活的语料库。

本文设计以英语为基础,使用了完整、丰富的MSCOCO-2014数据集。有八万多张图片作为训练集,另有四万多张图片作为测试集,且均带有相应的语义标注。这样完整的数据集更贴近一般语言环境,可以得到更好的训练效果。

\subsubsection{基础依赖包}
目前,常用的NLP算法在python、JAVA等语言中里面都有依赖包可以使用。

在我使用的python语言3.7版本中,可以使用依赖包nltk(Natural Language Toolkit)\upcite{nltk},这个包实现了超过五十种自然语言处理的基本算法,可以进行分词、语义角色解析等。

\section{本文工作目标}

\subsection{设计大纲构思}
本次设计需要实现如下功能,并达到要求:
\begin{enumerate}[fullwidth,itemindent=2em,label=\arabic*.]
    %\setlength{\parindent}{4em}
    \item 实现图像生成语义算法,并达到可用的效果;
    \item 实现语义生成图像算法,并且图像对一般人清晰可识别;
    \item 设计一个有机统一的系统,功能简洁易用。
\end{enumerate}

\subsection{本文篇章结构}

本文共分为五个章节。

第一章,介绍了研究的背景,并通过阅读文献,整理了相关的工作,从中找出比较适合于本设计的工作,并加以总结。

第二章,详细介绍了本设计所使用技术的详情,并说明了选取技术原因。

第三章,设计了实验的方案,通过一些前人制作好的开源项目,加以修改、总结,设计系统。

第四章,详细记述了实验的结果,并通过测评手段进行测评,分析了实验结果。

第五章,阐述了设计的意义,并且展望了未来系统进一步扩展迭代的方向,指明了系统的局限性。

\section{本章小结}
本章介绍了设计相关领域的需求背景,也全面地介绍了这一设计的意义与前景。接下来介绍了涉及到的相关技术发展现状和相关领域的工作,也说明了其中最适合于本设计的技术及其原因。最后说明了本设计的基本要求和构思,另外设计了篇章结构,也设定了工作的目标。