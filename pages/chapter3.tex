% Chapter 2

\chapter{方案设计}

\section{概要}
软件设计主界面使用python的tkinter包来制作。其中,图片转化为

\section{软件模型}
软件分为三个部分,第一部分是软件界面,需要实现简洁明了的操作功能,方便使用。

第二部分是图片标注功能。这一功能通过image caption算法

\section{变量符号与定义}
文中使用集合如下:
\begin{enumerate}[fullwidth,itemindent=2em,label=\arabic*.]
    \item $O$代表物品(object)的集合;
    \item $C$代表物品目录(catagory)的集合;
    \item $R$代表关系(relationship)的集合;
    \item 
\end{enumerate}

文中脚标定义如下:
\begin{enumerate}[fullwidth,itemindent=2em,label=\arabic*.]
    \item ${}_t$代表LSTM模型中的时序,;
    \item 
\end{enumerate}

文中小写字母对象定义如下:
\begin{enumerate}[fullwidth,itemindent=2em,label=\arabic*.]
    \item $O$代表物品(object)的集合;
    \item 
\end{enumerate}

文中主要函数定义如下:
\begin{enumerate}[fullwidth,itemindent=2em,label=\arabic*.]
    \item $\phi$代表物品(object)的集合;
    \item $\sigma$代表LSTM模型中的$sigmoid$函数;
    \item 
\end{enumerate}

\section{软件设计}

\subsection{图片标注方法}
\subsubsection{算法整体模型}
在这一模型中,要使用LSTM模型进行训练,训练出的模型放在主函数的调用函数中,实现图片翻译为自然语言的功能。

这一模型主要分为

\begin{algorithm}[H]
    \setstretch{1.5} % 代码间行距设定
    \SetAlgoLined
    \KwResult{Write here the result }
     initialization\;
     \While{While condition}{
      instructions\;
      \eIf{condition}{
       instructions1\;
       }{
       instructions3\;
      }
     }
    \caption{How to write algorithms}
  \end{algorithm}
\subsection{自然语言生成图片方法}
这一模型主要是用通过场景生成图像的GAN模型来训练,得到的模型放在主函数的调用函数中,实现自然语言生成图片的功能。

\begin{algorithm}[H]
    \setstretch{1.5} % 代码间行距设定
    \SetAlgoLined
    \KwResult{Write here the result }
     initialization\;
     \While{While condition}{
      instructions\;
      \eIf{condition}{
       instructions1\;
       }{
       instructions3\;
      }
     }
    \caption{How to write algorithms}
  \end{algorithm}
\section{本章小结}
本章具体地介绍了这一设计中所包含的设计细节及其理论依据,并确定了软件构架的结构,为完成代码构建和实验设计打下了基础。

本章中,确定了以python语言的tkinter库来写界面,保证界面简单简介易懂;用CNN和LSTM技术,基于注意力的方法,通过编码-解码的流程来完成图片标注功能;用基于位置场景的方法,由GNN和两个判别器构成的GAN模型来分步生成图片,最后生成合成图片的方法来实现由文本生成图片的功能。

在下一章里,我会继续记述实验的过程及其表现。
%{\songti \bfseries 宋体加粗} {\textbf{English}}

%{\songti \itshape 宋体斜体} {\textit{English}}

%%%{\songti \bfseries \itshape 宋体粗斜体} {\textbf{\textit{English}}}

%\section{编译}
%本模板必须使用XeLaTeX + BibTeX编译,否则会直接报错。 本模板支持多个平台,结合sublime/vscode/overleaf都可以使用。