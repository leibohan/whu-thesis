% Chapter 4

\chapter{实验与分析}
\section{实验环境}
\subsection{模型训练环境}
服务器环境,操作系统为
Ubuntu 16.04.6 LTS (GNU/Linux 4.4.0-142-generic x86\_64),使用命令行ssh工具登入。

内存64GB,搭配显卡2080 Ti 四块,由于资源分配原因本设计中实验使用两块。

使用anaconda 创建虚拟环境,为算法各自安装需要的依赖包与训练环境。
\subsection{软件样本运行环境}
Macbook Pro电脑 15.6寸屏幕版,mid-15 release,A1398型号;
OS X 10.12(Serrina High)系统;
集成显卡,内存16G,没有特别要求。

\section{图片标注实验}
\subsection{模型代码}
模型主题使用了开源代码,由第三方代码作者完成,在Github网站上开源;同时我在其中做了修缮维护,保证了复现在当前版本的依赖包上和现在可以运行的系统里仍然可以运行。

原代码开发环境是python 2.7和tensorflow 0.14,由于目前科学计算使用的python以python3为主流,我将其修改维护至了python 3.6和tensorflow 1.14版本下可以运行的代码。

代码结构分为几个部分:主函数、基本模型类、训练算法。
\subsection{运行过程}
通过在论坛上调研、向专家咨询,我认为较少的数据会导致图片标注模型的效果减弱,会限定构图、限定主题、限定语汇,且识别精确度会很差。为此,我选用了MSCOCO数据集作为训练和测试集。对于这次实验,我先选择了最早的COCO2014数据集,作为试验,后来发现训练时间成本较长,决定使用COCO2014数据集下训练生成的模型作为应用模型。

这一算法的虚拟实验环境python版本为3.7,tensorflow版本为1.14.0,其中安装的依赖包有:numpy 1.17.2,OpenCV 4.1.1.26,NLTK 3.4.5,Pandas 0.25.1,Matplotlib 3.1.1,tqdm 4.36.1。
其中NLTK用作训练集中文字信息的分词处理,仅仅执行了nltk.download("punkt"),作为数据支撑,没有下载完整数据。

模型的训练在实验环境下使用4块NVIDIA 2080 Ti显卡,训练了233小时,执行了100个epoch,每一个epoch都遍历了训练集中的所有数据。


\subsection{遇到的主要问题与解决方法}
\subsubsection{数据传输问题}
服务器的物理地址在浙江大学,属于教育网的浙江位置,从我的工作地点到浙江大学校内当前跳转节点较多,使用RVPN进行连接后,数据传输速度上限是2.00Mbits/sec,所以传输一个总大小为19GiB以上的数据集,需要的时间为40000秒,相当于十余小时。另外,由于RVPN的连接不稳定,而scp命令传输文件不支持断点续传,所以经常出现文件损坏的现象。

经过多次尝试和实践,我采用分包的方式,将数据集化为1GiB到2GiB大小的压缩包,由scp命令逐一传输;同时使用完善、成熟的数据集,尽量最大化一次成功率。

\subsubsection{依赖包函数不兼容问题}
tensorflow在目前的稳定版本中,已经增加了很多新的借口,而1版本和0版本的接口,有一些已经取消了。即使我安装的是1版本的tensorflow环境,仍然有一些接口无法调用。经过调研,我使用了评价较高的tensorflow.compat.v1库,来实现变易出错的接口,解决了大部分问题。

\section{自然语言生成图片实验}
模型使用了开源代码,由论文原作者完成,在Github网站上开源;同时我在其中微调,以适合我所需的数据集。

\section{实验分析与总结}


\footnote{在github网址 https://github.com/hanzhanggit/StackGAN-Pytorch 可以找到预先训练完成的模型,并从原文找到了评测数据}

\section{本章小结}
这一章节分别介绍了软件界面编码方式、软件两个功能模型训练方式过程与软件部署调试的过程。在软件界面编码中,试用了tkinter进行编码,用比较简明的方式,实现了用例图中的效果;在软件功能模型训练中,记述了LSTM模型和GAN模型分别在环境安装、数据集选取、训练次数调整和呈现方式的过程

%{\songti \bfseries 宋体加粗} {\textbf{English}}

%{\songti \itshape 宋体斜体} {\textit{English}}

%%%{\songti \bfseries \itshape 宋体粗斜体} {\textbf{\textit{English}}}

%\section{编译}
%本模板必须使用XeLaTeX + BibTeX编译,否则会直接报错。 本模板支持多个平台,结合sublime/vscode/overleaf都可以使用。